%%==============================================================
%% Modelo de TCC para o curso de Sistemas de Informação
%% da Universidade Federal de Viçosa - Campus de Rio Paranaíba
%% Autor: Rodrigo Smarzaro (smarzaro@ufv.br)
%% Última versão Março/2014
%% Arquivo em formato UTF-8
%% Compilar com pdftex
% %Precisa do arquivo UFV.sty
%%==============================================================

\documentclass[
	% -- opções da classe memoir --
	12pt,				    % tamanho da fonte
	openright,			    % capítulos começam em pág ímpar (insere página vazia caso preciso)
	oneside,			    % para impressão só no anverso. Oposto a twoside
	a4paper,			    % tamanho do papel.
    % -- opções do pacote abntex2 --
    % chapter=TITLE,         % Títulos em maiúsculas
    sumario=tradicional,    % Sumário padrão memoir (mais bonito "imo")
    % -- opções do pacote babel --
	english,			    % idioma adicional para hifenização
	brazil,				    % o último idioma é o principal do documento
	]{abntex2}              % Personaliza a capa. Precisa do arquivo ufv.cls para funcionar.



% Pacotes fundamentais
\usepackage{abntex2-UFV}        % Personalização para a Universidade Federal de Viçosa
\usepackage{lmodern}			% Usa a fonte Latin Modern			
\usepackage[T1]{fontenc}		% Selecao de codigos de fonte de saída
\usepackage[utf8]{inputenc}		% Codificacao do documento (conversão automática dos acentos)
\usepackage{indentfirst}		% Indenta o primeiro parágrafo de cada seção.
\usepackage{graphicx}			% Inclusão de gráficos
\usepackage{booktabs}           % \toprule, \midrule e \bottomrule para tabelas
% Sistema autor-data com títulos nas referências em negrito
\usepackage[alf,abnt-emphasize=bf]{abntex2cite}	


% ---
% CONFIGURAÇÕES DE PACOTES
% ---

% Informações de dados para CAPA e FOLHA DE ROSTO
\titulo{Template para Trabalhos de Conclusão de Curso da UFV}
\autor{Rodrigo Smarzaro}
\local{Rio Paranaíba}
\data{2014}
\orientador{Nome do Orientador}    % redefinido no abntex2-UFV para aceitar Instituição (default = UFV-CRP)
%\coorientador{Nome do Coorientador}
\instituicao{Universidade Federal de Viçosa}

\campus{\emph{Campus} de Rio Paranaíba}      % pacote abntex2-UFV
\curso{Sistemas de Informação}               % pacote abntex2-UFV
\membrobancaA{Membro da Banca A}             % pacote abntex2-UFV default = UFV-CRP
\membrobancaB[UFMG]{Membro da Banca B}       % pacote abntex2-UFV default = UFV-CRP
\databanca{\today}                           % pacote abntex2-UFV

% O preambulo deve conter o tipo do trabalho, o objetivo,
% o nome da instituição e a área de concentração
\preambulo{Monografia apresentada à Universidade Federal de Viçosa, como parte das exigências para a a aprovação na disciplina Trabalho de Conclusão de Curso I}
% ---

% ---
% Configurações de aparência do PDF final

% informações para o arquivo pdf de saída
% Interessante alterar a cor dos links para preto(black)
% para imprimir
\makeatletter
\hypersetup{
        % metadados
		pdftitle={\@title},
		pdfauthor={\@author},
    	pdfsubject={\imprimirpreambulo},
	    pdfcreator={LaTeX with abnTeX2},
		colorlinks=true,   % false: links em frame; true: links coloridos
    	linkcolor=black,    % cor dos links no documento
    	citecolor=blue,    % cor dos links para a bibliografia
    	filecolor=magenta, % cor dos links para arquivos
		urlcolor=blue,     % cor dos links para sites
		bookmarksdepth=4   % profundidade do sumário do PDF
}
\makeatother
% ---

\begin{document}
% Retira espaço extra obsoleto entre as frases.
\frenchspacing

% ----------------------------------------------------------
% ELEMENTOS PRÉ-TEXTUAIS
% ----------------------------------------------------------
\pretextual

% Capa
\imprimircapa

% Folha de rosto
\imprimirfolhaderosto
% ---

% Inserir folha de aprovação
\imprimirfolhadeaprovacao

% Dedicatória
\begin{dedicatoria}
   \vspace*{\fill}
   \centering
   \noindent
   \textit{Texto qualquer da dedicatória}
   \vspace*{\fill}
\end{dedicatoria}
% ---

% Agradecimentos
\begin{agradecimentos}

\end{agradecimentos}
 ---

% Epígrafe
\begin{epigrafe}
    \vspace*{\fill}
	\begin{flushright}
		\textit{``Word? nunca mais.''\\
		(Qualquer usuário de \LaTeX)}
	\end{flushright}
\end{epigrafe}
% ---

% RESUMOS

% resumo em português
\begin{resumo}
 \noindent
%Insira o resumo aqui

 \vspace{\onelineskip}

 \noindent
 \textbf{Palavras-chaves}: TCC, abntex, LaTeX, UFV.
\end{resumo}

% resumo em inglês
%\begin{resumo}[Abstract]
% \begin{otherlanguage*}{english}
%   \noindent
%   % Insira o abstract aqui
%
%   \vspace{\onelineskip}
%
%   \noindent
%   \textbf{Key-words}: TCC, latex, abntex, UFV..
% \end{otherlanguage*}
%\end{resumo}

% inserir lista de ilustrações
\pdfbookmark[0]{\listfigurename}{lof}
\listoffigures*
\cleardoublepage
% ---

% inserir lista de tabelas
\pdfbookmark[0]{\listtablename}{lot}
\listoftables*
\cleardoublepage
% ---


% Lista de siglas e abreviaturas (opcional)
% sintaxe: \item [sigla] Descrição da sigla

%\begin{siglas}
%\item[ABNT] Absurdas Normas Técnicas
%\item[UFV] Universidade Federal de Viçosa
%\item[CRP] \emph{Campus} de Rio Paranaíba
%\end{siglas}

% Lista de símbolos (opcional)
% sintaxe: \item [simbolo] Descrição do símbolo

%\begin{simbolos}
%\item[$\infty$ ] Infinito
%\end{simbolos}


% inserir o sumario
\pdfbookmark[0]{\contentsname}{toc}
\tableofcontents*
\cleardoublepage
% ---



% ----------------------------------------------------------
% ELEMENTOS TEXTUAIS
% ----------------------------------------------------------
\textual


%Modifique a estrutura dos capítulos e seções de acordo com a necessidade do seu trabalho
\chapter{Introdução}\label{sec:introducao}
Alguns links interessantes para se trabalhar com a classe abn\TeX\ e \LaTeX\ em geral\footnote{E também para usar alguns comandos de citação como exemplo}:
\begin{alineas}
  \item Informações da classe Abn\TeX : \citeonline{abntex2classe}
  \item Ajustes nas citações e referências: \citeonline{abntex2cite} e \citeonline{abntex2cite-alf}
  \item Classe memoir (base do Abn\TeX\ ): \apudonline{memoir}{abntex2classe}
  \item Livros interessantes sobre \LaTeX: \cite{Dongen2012,LeslieLamport90,FrankMittelbach111,Dongen2012}
  \item Distribuição \LaTeX\ para windows: \url{http://miktex.org/}
  \item Editor \LaTeX\ gratuito: \url{http://texstudio.sourceforge.net/}
  \item Gerenciador de arquivos \texttt{.bib}: \url{http://jabref.sourceforge.net/}
  \item Gerenciador de artigos: \url{http://www.mendeley.com/}
  \item Exemplo de Tabela: Veja \autoref{tab:cronograma}

\end{alineas}
\chapter{Objetivos}\label{sec:objetivos}

\section{Objetivos Específicos}\label{sec:ObjetivosEspec}

\chapter{Referencial Teórico}\label{sec:RefTeorico}

\section{Trabalhos Relacionados}\label{sec:TrabRel}

\chapter{Métodos}\label{sec:metodos}
\begin{figure}[htbp]
  \begin{center}
  \includegraphics[width=.5\linewidth]{logoufv}\\
  \end{center}
  \caption[Exemplo de Figura]{Exemplo de inserção de figura no \LaTeX. A legenda deve vir abaixo da figura. Pode usar o comando \texttt{\textbackslash legend} ou \texttt{\textbackslash fonte} para inserir a fonte da figura. Observe que na lista de ilustrações foi utilizado o nome curto fornecido como parâmetro do caption da figura (veja o arquivo fonte .tex) ao invés dessa legenda estupidamente extensa feita de forma proposital}
  \label{fig:logo}
  \legend{Fonte: Próprio Autor}
\end{figure}

\chapter{Resultados Esperados}\label{sec:resultEsperados}

\chapter{Cronograma}\label{sec:cronograma}
O abn\TeX\ introduziu o comando \texttt{IBGEtab} para formatação de tabelas. Um exemplo de tabela convencional do \LaTeX\ pode ser observado na \autoref{tab:cronograma} enquanto um exemplo usando o \texttt{IBGEtab} é mostrado na Tabela \ref{tab:cronogramaIBGE}.

\begin{table}[htbp]
  \centering
    \caption[Cronograma Normal]{Cronograma do Projeto em Meses}
    \label{tab:cronograma}
    \begin{tabular}{lcccccccccccc} %|c|c|c|c|c|c|c|c|c|c|c|c
    \toprule
    \textbf{Atividade} & \textbf{1} & \textbf{2} & \textbf{3} & \textbf{4} & \textbf{5} & \textbf{6} & \textbf{7} & \textbf{8} & \textbf{9} & \textbf{10} & \textbf{11} & \textbf{12} \\
    \midrule
        Revisão Bibliográfica & $\bullet$ & $\bullet$ & & & & & & & & & & \\
        Métodos & & & $\bullet$ & $\bullet$ & & & & & & & & \\
        Testes & & & & $\bullet$ & $\bullet$ & $\bullet$ & & & & & & \\
        Resultados & & & & & & & $\bullet$ & $\bullet$ & & & & \\
        Conclusão & & & & & & & $\bullet$ & $\bullet$ & $\bullet$ & & & \\
        Banca & & & & & & &&&& $\bullet$ & $\bullet$ & $\bullet$ \\
    \bottomrule
    \end{tabular}%
    \fonte{Próprio Autor}
\end{table}%



\begin{table}[htbp]
    \IBGEtab{
    \caption[Cronograma (IBGE)]{Cronograma do Projeto em Meses usando o comando IBGEtab para a formatação da tabela}
    \label{tab:cronogramaIBGE}
    }{
    \begin{tabular}{lcccccccccccc} %|c|c|c|c|c|c|c|c|c|c|c|c
    \toprule
    \textbf{Atividade} & \textbf{1} & \textbf{2} & \textbf{3} & \textbf{4} & \textbf{5} & \textbf{6} & \textbf{7} & \textbf{8} & \textbf{9} & \textbf{10} & \textbf{11} & \textbf{12} \\
    \midrule
        Revisão Bibliográfica & $\bullet$ & $\bullet$ & & & & & & & & & & \\
        Métodos & & & $\bullet$ & $\bullet$ & & & & & & & & \\
        Testes & & & & $\bullet$ & $\bullet$ & $\bullet$ & & & & & & \\
        Resultados & & & & & & & $\bullet$ & $\bullet$ & & & & \\
        Conclusão & & & & & & & $\bullet$ & $\bullet$ & $\bullet$ & & & \\
        Banca & & & & & & &&&& $\bullet$ & $\bullet$ & $\bullet$ \\
    \bottomrule
    \end{tabular}%
    }{
    \fonte{Próprio Autor}}
\end{table}%



% ----------------------------------------------------------
% ELEMENTOS PÓS-TEXTUAIS
% ----------------------------------------------------------
\postextual

% Referências bibliográficas

\bibliography{referencias}

% Caso sejam necessários apêndices ou anexos em seu documento
% Use os ambientes abaixo

%% Apêndices
%
%% Inicia os apêndices
%\begin{apendicesenv}
%
%% Imprime uma página indicando o início dos apêndices
%\partapendices
%
%\chapter{Primeiro Apêndice}
%
%\chapter{Segundo Apêndice}
%
%\end{apendicesenv}
%
%
%% ----------------------------------------------------------
%% Anexos
%% ----------------------------------------------------------
%\begin{anexosenv}
%
%% Imprime uma página indicando o início dos anexos
%\partanexos
%
%\chapter{Primeiro Anexo}
%\lipsum[30]
%
%\chapter{Segundo Anexo}
%\lipsum[31]
%
%\end{anexosenv}

\end{document}
